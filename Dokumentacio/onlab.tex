\documentclass[11pt,a4paper,oneside]{report}             % Single-side
%\documentclass[11pt,a4paper,twoside,openright]{report}  % Duplex

%\PassOptionsToPackage{chapternumber=Huordinal}{magyar.ldf}
\usepackage{t1enc}
\usepackage[latin2]{inputenc}
\usepackage{amsmath}
\usepackage{amssymb}
\usepackage{enumerate}
\usepackage[thmmarks]{ntheorem}
\usepackage{graphics}
\usepackage{epsfig}
\usepackage{listings}
\usepackage{color}
%\usepackage{fancyhdr}
\usepackage{lastpage}
\usepackage{anysize}
\usepackage[magyar]{babel}
\usepackage{sectsty}
\usepackage{setspace}  % Ettol a tablazatok, abrak, labjegyzetek maradnak 1-es sorkozzel!
\usepackage[hang]{caption}
\usepackage{hyperref}

%--------------------------------------------------------------------------------------
% Main variables
%--------------------------------------------------------------------------------------
\newcommand{\vikszerzo}{B�dis-Szomor� Andr�s}
\newcommand{\vikkonzulens}{dr.~Konzulens Elem�r}
\newcommand{\vikcim}{Elektronikus terel�k}
\newcommand{\viktanszek}{M�r�stechnika �s Inform�ci�s Rendszerek Tansz�k}
\newcommand{\vikdoktipus}{Diplomaterv}
\newcommand{\vikdepartmentr}{B�dis-Szomor� Andr�s}

%--------------------------------------------------------------------------------------
% Page layout setup
%--------------------------------------------------------------------------------------
% we need to redefine the pagestyle plain
% another possibility is to use the body of this command without \fancypagestyle
% and use \pagestyle{fancy} but in that case the special pages
% (like the ToC, the References, and the Chapter pages)remain in plane style

\pagestyle{plain}
%\setlength{\parindent}{0pt} % �ttekinthet�bb, angol nyelv� dokumentumokban jellemz�
%\setlength{\parskip}{8pt plus 3pt minus 3pt} % �ttekinthet�bb, angol nyelv� dokumentumokban jellemz�
\setlength{\parindent}{12pt} % magyar nyelv� dokumentumokban jellemz�
\setlength{\parskip}{0pt}    % magyar nyelv� dokumentumokban jellemz�

\marginsize{35mm}{25mm}{15mm}{15mm} % anysize package
\setcounter{secnumdepth}{0}
\sectionfont{\large\upshape\bfseries}
\setcounter{secnumdepth}{2}
\singlespacing
\frenchspacing

%--------------------------------------------------------------------------------------
%	Setup hyperref package
%--------------------------------------------------------------------------------------
\hypersetup{
    bookmarks=true,            % show bookmarks bar?
    unicode=false,             % non-Latin characters in Acrobat�s bookmarks
    pdftitle={\vikcim},        % title
    pdfauthor={\vikszerzo},    % author
    pdfsubject={\vikdoktipus}, % subject of the document
    pdfcreator={\vikszerzo},   % creator of the document
    pdfproducer={Producer},    % producer of the document
    pdfkeywords={keywords},    % list of keywords
    pdfnewwindow=true,         % links in new window
    colorlinks=true,           % false: boxed links; true: colored links
    linkcolor=black,           % color of internal links
    citecolor=black,           % color of links to bibliography
    filecolor=black,           % color of file links
    urlcolor=black             % color of external links
}

%--------------------------------------------------------------------------------------
% Set up listings
%--------------------------------------------------------------------------------------
\lstset{
	basicstyle=\scriptsize\ttfamily, % print whole listing small
	keywordstyle=\color{black}\bfseries\underbar, % underlined bold black keywords
	identifierstyle=, 					% nothing happens
	commentstyle=\color{white}, % white comments
	stringstyle=\scriptsize\sffamily, 			% typewriter type for strings
	showstringspaces=false,     % no special string spaces
	aboveskip=3pt,
	belowskip=3pt,
	columns=fixed,
	backgroundcolor=\color{lightgray},
} 		
\def\lstlistingname{lista}	

%--------------------------------------------------------------------------------------
%	Some new commands and declarations
%--------------------------------------------------------------------------------------
\newcommand{\code}[1]{{\upshape\ttfamily\scriptsize\indent #1}}

% define references
\newcommand{\figref}[1]{\ref{fig:#1}.}
\renewcommand{\eqref}[1]{(\ref{eq:#1})}
\newcommand{\listref}[1]{\ref{listing:#1}.}
\newcommand{\sectref}[1]{\ref{sect:#1}}
\newcommand{\tabref}[1]{\ref{tab:#1}.}

\DeclareMathOperator*{\argmax}{arg\,max}
%\DeclareMathOperator*[1]{\floor}{arg\,max}
\DeclareMathOperator{\sign}{sgn}
\DeclareMathOperator{\rot}{rot}
\definecolor{lightgray}{rgb}{0.95,0.95,0.95}

\author{\vikszerzo}
\title{\viktitle}
\includeonly{
	guideline,%
	project,%
	titlepage,%
	declaration,%
	abstract,%
	introduction,%
	chapter1,%
	chapter2,%
	chapter3,%
	acknowledgement,%
	appendices,%
}
%--------------------------------------------------------------------------------------
%	Setup captions
%--------------------------------------------------------------------------------------
\captionsetup[figure]{
%labelsep=none,
%font={footnotesize,it},
%justification=justified,
width=.75\textwidth,
aboveskip=10pt}

\renewcommand{\captionlabelfont}{\small\bf}
\renewcommand{\captionfont}{\footnotesize\it}

%--------------------------------------------------------------------------------------
% Table of contents and the main text
%--------------------------------------------------------------------------------------
\begin{document}
\singlespacing
\include{guideline}
\include{project}

\pagenumbering{arabic}
\onehalfspacing
\tableofcontents\vfill

\chapter{Introduction}

Smartphones have become a defining part of our days. By now we can't even imagine our everyday life without them. We read e-mails, browse the Internet or monitor daily exchange rates. Now we can do all important things with our phones. Listening to music and watching series, but in terms of photography and video capabilities, they are not far behind cameras designed for professional purposes. Their computing capacity has grown exponentially over the years. The quality of camera systems is also undergoing significant progress every year.Perhaps this is why many people say that the development of telephones is starting to slow down. This industry has reached its peak.

At the time, in 2007, the iPhone was considered a huge world sensation and now we know it was a world-changing announcement. At the time of writing this report, there are several reports that Apple is working on the introduction of a similarly large and decisive new product line. The so-called Apple AR/VR glasses. Only time will tell if this really happened until then I thought that it would be a great advance if I started familiarising with this technology. If this will be the next "touchscreen phone era", I thought it would be worthwhile to start dealing with it in time. Regardless, I have long been interested in the technology's operating principle and scope of use. From the consumer side, what kind of application possibilities does augmented reality technology have? At that time, I wanted to implement a similar augmented reality-like navigation system in my own high school navigation application. Unfortunately, I did not know about the existence of these technologies at the time.

As I mentioned in the first paragraph, today's phones have a huge computing capacity. This is also why it is possible for such augmented reality applications to run on our phones. Based on the 1 or better case 2 or 3 lenses placed in the phone, it can measure the 3D depth of each scenario. There are phones, such as the iPhone 13 Pro, which have a laser LiDAR sensor placed specifically for this purpose. In the 2010s, many devices with similar technology came out, such as Microsoft's Kinect developed for the X-Box 360 console. It is also worth mentioning Google Glass developed by Google or HoloLens and HoloLens 2 marketed by Microsoft. Unfortunately, the former mentioned Google Glass was not an undivided success and in 2023 the sale of the glasses was discontinued.

Last but not least, it is important to mention the Oculus Quest 2 VR headset introduced by Meta and the computer use and games based on it. In addition, the Metaverse, announced by Meta 2 years ago, is also an important milestone in the life of virtual reality and therefore also in the life of augmented reality.

As the examples above clearly reflect, it will be an important and presumably defining technology in the future. In light of this, during my self-lab, I got to know the ARKit and RealityKit developed and used by Apple. And with the help of the frameworks, I developed an augmented reality application displaying economic metrics. And during the report I used LaTex, because I felt it was time to familiarize myself with this kind of documentation language.

\chapter{Current techologies}

Before I started the development of the application I have done a research period to have a better overview of the current market. There are several frameworks to choose from to start developing augmented reality applications. The most used AR frameworks are Google ARCore, Apple ARKit and RealityKit, Simple CV and Unity AR Foundation.
I ended up using Apple ARKit to build my app. I've been working with iOS development for several years and I'm sure with the Swift language as well.



%\listoffigures\addcontentsline{toc}{chapter}{�br�k jegyz�ke}
%\listoftables\addcontentsline{toc}{chapter}{T�bl�zatok jegyz�ke}

\bibliography{mybib}
\addcontentsline{toc}{chapter}{Irodalomjegyz�k}
\bibliographystyle{plain}

\include{appendices}

\label{page:last}
\end{document}
