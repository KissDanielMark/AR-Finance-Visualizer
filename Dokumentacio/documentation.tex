\documentclass{report}

\input{preamble}

\title{\Huge{MSc - Önálló laboratórium 1}\\Augmenteted reality using ARKit}
\author{\huge{Daniel Mark Kiss}}
\date{2023}

\begin{document}

\maketitle
\newpage
\tableofcontents
\pagebreak

\chapter{Introduction}



\chapter{Current techologies}

Before I started the development of the application I have done a research period to have a better overview of the current market. There are several frameworks to choose from to start developing augmented reality applications. The most used AR frameworks are Google ARCore, Apple ARKit and RealityKit, Simple CV and Unity AR Foundation.

\section{Apple ARKit and RealityKit}


\section{Unity}

One of the main frameworks that we can use if we want to develope augmented reality application is Unity.
Unity is a purpose-build framework for augmented reality development. It's key feature is that with the framework developer can easily deploy it's product across multiple mobile and wearable AR devices. It includes core features from ARKit(Apple), ARCore(Google) and HoloLens(Microsoft) as well unique features.

AR Foundation which is the recommended framework for Unity development there are several features what the developers can use like:
\begin{itemize}
    \item Device tracking
    \item Raycast
    \item Plane detection
    \item Gestures
    \item Face tracking
\end{itemize}


\chapter{Development}

\section{Overview}
For the development phase of the project I used Apple's XCode integrated developer environment(IDE) as the main developer platform for iOS and ARKit development. For testing I used an iPhone 11 with dual camera system.
I have also version controled the whole development process using git and publishing it on GitHub. Not only the source code can be found there but also the documentation of this project as I have writen it using \LaTeX{}.

\section{Yahoo Rest API}

For retriving the displayed informations I used Yahoo's open and available for free financial API.

\chapter{Presentation of finished work}

In this section I will present and showcase my finished appliction. I will include screenshots to have a better representation and understanding for the reader.


\chapter{Sources}

\section{External Links}

\hyperlink{https://dynamics.microsoft.com/en-us/mixed-reality/guides/what-is-augmented-reality-ar/}{Microsoft - What is Augmenteted Reality}

\hyperlink{https://developer.apple.com/augmented-reality/}{Apple - Augmented Reality}

\hyperlink{https://developer.apple.com/documentation/realitykit/}{Apple - RealityKit}

\hyperlink{https://www.kodeco.com/books/apple-augmented-reality-by-tutorials/v1.0/chapters/iii-introduction}{Kodeco - RalityKit tutorials}

\hyperlink{https://developer.apple.com/forums/thread/658300}{Rövid leírás - https://developer.apple.com/forums/thread/658300}

\hyperlink{https://developer.apple.com/documentation/realitykit/adding-procedural-assets-to-a-scene}{Apple - Adding procedural assets to a scene}

\hyperlink{https://coledennis.medium.com/tutorial-generating-3d-text-with-realitykit-in-a-swiftui-app-fa2a50403012}{Medium - Adding 3D text to scene}

\hyperlink{https://betterprogramming.pub/take-an-arview-snapshot-in-realitykit-93b620cf99b3}{BetterProgramming - Taking AR view snapshot}

\hyperlink{https://www.youtube.com/watch?v=9R_G0EI-UoI}{YouTube - Placing models}

\hyperlink{https://github.com/maxxfrazer/FocusEntity}{GitHub - FocusEntity}

\hyperlink{https://betterprogramming.pub/how-to-add-text-to-an-arview-in-an-ios-application-tutorial-f3f746f4dc1f}{BetterProgramming - Update model entity}


\end{document}
